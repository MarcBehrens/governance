
\documentclass[a4paper, 11pt]{article}
\usepackage[ascii]{inputenc}
\usepackage{supertabular}
\usepackage[ngerman]{babel}
\usepackage{amsmath}
\usepackage{amssymb,amsfonts,textcomp}
\usepackage {geometry}
\geometry{a4paper,top=25mm,left=30mm,right=25mm,bottom=30mm}
\usepackage{color}
\usepackage{array}
\usepackage{hhline}
\usepackage{hyperref}
\hypersetup{colorlinks=true, linkcolor=blue, citecolor=blue, filecolor=blue, urlcolor=blue}


\begin{document}
{\begin{center}\huge\bf openETCS PMB Meeting Minutes\end{center}}
%{\begin{center}\huge\bf Project Management Board (PMB)\end{center}}
\section{Meeting Information}

\renewcommand{\arraystretch}{1.5}
\begin{supertabular}{m{.2\textwidth}m{.8\textwidth}}
%\hline
Subject & PMB Weekly Scrum: Project Management Board\\
Date \& time & 2013-04-26, 13:30h--14:30h\\
Location & Telco and Goto-Meeting\\
Called up by & Klaus-R\"udiger Hase\\
Participants & PMB,
Sylvain Baro,
%Benjamin Beichler,
%Marc Behrens,
%Cecile Braunstein,
%Stephane Callet,
%Cyril Cornu, 
%Gilles Dalmas,
%Jens Gerlach, 
%Frank Golatowski, 
Klaus-R\"udiger Hase,
Bernd Hekele,
%Jonas Helming,
Baseliyos Jacob,
Michael Jastram, 
Bego\~na Laibarra,
Peter Mahlmann, 
%Jaime Paniagua,
%Marielle Petit-Doche, 
%Stan Pinte,
Stefan Rieger,
%Uwe Steinke,
%Jo\~ao Santos,
%Jan Welte,
Giovanni Zanelli\\

Minutes by & Bernd Hekele\\

%\hline
\end{supertabular}
\renewcommand{\arraystretch}{1.0}

%\line(1,0){440}

\section{{Agenda}}
\begin{itemize}
\item Proposal for new meeting schedule of weekly scrum meetings
\item FPP and PCA release
\item Update on SRSS task
\item Next PCC Meeting
\item Review Process
\item A.O.B.
\end{itemize}

\section{Discussion}
\begin{itemize}
\item Proposal for new meeting schedule of weekly scrum meetings\
We see some space for optimisation in our weekly Scrum meetings. The proposal described below has been accepted by the PMB.

The WP- meetings will be shortened to last 15 minutes each (again, strictly timeboxed). This slot is seen sufficient for reporting on progress and impediments. 

If a special topic pops up we have sufficient time before and after the weekly meetings to meet on demand. The goto-meeting infrastructure will be made available on request. If you need the platform, make sure you address directly all persons involved.

We still have weekly meetings on Friday. We have time before 10:00 and after 12:15 for special topics.
We skip our German cluster. This gives the following schedule:
\begin{itemize}
\item 8:00 - 10:00	Topics on demand
\item 10:00 - 10:15	WP6
\item 10:15 - 10:30	WP5
\item 10:30 - 10:45	WP4
\item 10:45 - 11:00	WP3
\item 11:00 - 11:15	WP7
\item 11:15 - 11:30	WP2
\item 11:30 - 12:00	PMB
\item 12:00 - 12:15	WP1
\item 13:00 - 16:00	Topics on demand
\end{itemize}
The meeting owners are invited to set-up their meeting in the new google project calendar.

\item FPP and PCA release\
Updated documents are out for final confirmation. Results are expected by Friday, May, 17th.

\item Update on SRSS task\
We are still collecting participants. Confirmation missing from SNCF and Alstom.

\item Next PCC Meeting\
The meeting originally scheduled in May is cancelled.

\item Review Process\
SQS has provided a new review process. This review process is currently in the state of a prototype. It is in use for the review of the document as such and for quality related documents. After confirmation, we plan to take this process in use. Till then the current agreements - especially the review process in use with requirement documents - can be used for ongoing activities.
The new process resp. quality document has to respect the review results which have been achieved beforehand.
It is foreseen to have a (live or goto-) meeting in order to finalize the review.

\item A.O.B.\
The planning for the July ITEA review is avaiolable in Github (repository management). In the plan the presentations are proposed. Each active WP is expected to contribute with a presentation. Please, have a look at the plan and check your participation.

Based on Google we take the project calendar now in use. Peyman has contributed in the ecosystem wiki a document on how to use the calendar:\\
\url{https://github.com/openETCS/ecosystem/wiki/How-to-use-the-openetcs-calendar}

\end{itemize}

\line(1,0){440}
\section{\underline{Notes}}

\end{document}
